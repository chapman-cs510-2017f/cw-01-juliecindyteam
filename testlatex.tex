%Cynthia Parks, Julie Gardner-Hoag
%2303535, 2299636
%cparks@chapman.edu, gardnerh@chapman.edu
%CS 510 Computing for Scientists
%CW 1
%Definition of derivative page to help practice latex commands and use of git
%and cocalc servers.
\documentclass[aps,pra,notitlepage,amsmath,amssymb,letterpaper,12pt]{revtex4-1}
\usepackage{amsthm}
\usepackage{graphicx}
%  Above uses the Americal Physical Society template for Physical Review A
%  as a reasonable and fully-featured default template

%  Below define helpful commands to set up problem environments easily
\newenvironment{problem}[2][Problem]{\begin{trivlist}
\item[\hskip \labelsep {\bfseries #1}\hskip \labelsep {\bfseries #2.}]}{\end{trivlist}}
\newenvironment{solution}{\begin{proof}[Solution]}{\end{proof}}

% --------------------------------------------------------------
%                   Document Begins Here
% --------------------------------------------------------------

\begin{document}

\title{Derivative}
\author{Julie Gardner-Hoag, Cynthia Parks}
\affiliation{CS 510, Schmid College of Science and Technology, Chapman University}
\date{\today}

\maketitle

\section{Abstract} % Purpose of the paper
We will briefly discuss the definition, equation, and results of derivatives.

\section{Derivative of a Function} % Specify main sections this way

% x.yz is the problem number
\begin{problem}{1}
  What is the Derivative of a Function
\end{problem}

\begin{solution} %You can also use proof in place of solution
The Derivative of a function measures change of the output value of the function with relation to change of the input value.  This can be seen with the following equation.
\begin{align}
f'(x)= \lim_{x \to +\infty} \frac{f(x+h) - f(x)}{h}\\
\nonumber
\end{align}
\subsection{Result} % Adding a result of differentiation
One important result derivatives give us is that if $f(x)$ is differentiable at $x=a$, then $f(x)$ is continuous at $x=a$.
% Use align environments for equations. The \\ is a newline character. The & is the alignment character.
% Using align* or \nonumber on each line removes equation numbers
\end{solution}

\subsection{Illustrating the Derivative of a Function} % Specify subsections and subsubsections this way

%Figures can be included easily.

\begin{figure}[h!] % h forces the figure to be placed here, in the text
  \includegraphics[width=0.4\textwidth]{derivative.png}  % if pdflatex is used, jpg, pdf, and png are permitted
  \caption{Graphing the derivative of f(x).}
  \label{fig:figlabel}
\end{figure}

The derivative of a function, illustrated by the green line in Figure 1, is the slope of the tangent line of f(x). As f(x) reaches a maximum or a minimum point on the graph, the value of f'(x) is 0. When the value of f(x) is decreasing, the value of f'(x) is negative. When f(x) is increasing, f'(x) is positive.
% Repeat as needed

\end{document}